\documentclass{article}
\usepackage[margin=1in]{geometry}
\usepackage{booktabs}
\usepackage{amsmath}
\usepackage[per-mode=symbol]{siunitx}
\usepackage[american,cuteinductors,siunitx]{circuitikz} % smartlabels, cuteinductors
\sisetup{list-final-separator = {, and }}

\author{Graham Keeth}
\title{Documentation for Power Supply Project}

\begin{document}
\maketitle
\section{Requirements}
\begin{itemize}
  \item{one dual-polarity output channel; i.e. one positive and one negative output equidistant from ground}
  \item{output voltage range: up to \SI{12}{\volt}}
  \item{current limit adjustable from \SIrange{0}{300}{\milli\ampere}}
  \item{seven-segment displays for current and voltage setpoints}
  \item{knob to set current and voltage limits}
  \item{output on/off switch}
  \item{indicator light for constant current or constant voltage. Perhaps this
          could be an LED indicating the output is on with a separate LED for
          constant current mode.}
\end{itemize}

\section{Components}
\begin{itemize}
  \item{voltage regulation is accomplished using an LM317 for the positive rail
    and an LM337 for the negative rail}
  \item{power input to the module is through a \SI{15}{\volt}, \SI{1}{\ampere}
    AC-to-AC power adapter}
\end{itemize}

\section{AC Input and Rectification}
\subsection{Schematic}
\begin{circuitikz}
    \draw
    (3,0) coordinate (N1)
        node[transformer core](x){}
        (x.base) node{10:1}
        (x.A2) to[short] ++(-1, 0)
        to[sV, l=$V_\mathrm{AC}$] ++(0, 2.1)
        to[short] (x.A1)
    (x.B1)
        to[short] ++(3, 0) coordinate (D1)
        to[D*] ++(2, 0)
        to[short] ++(1, 0) coordinate (C1top)
        to[short] ++(2, 0) coordinate (vposout)
        to[open, o-o, v^=$V_\mathrm{DC}$] ++(0, -2.1) coordinate (gndout)
        to[short] (x.B2)
    ++(2, 0)
        node[circ]{}
        node[ground]{}
    (gndout)
        to[open, o-o, v^=$V_\mathrm{DC}$] ++(0, -2.1) coordinate (vnegout)
        to[short] ++(-2, 0) coordinate (C2bot)
        to[short] ++(-1, 0)
        to[D*] ++(-2, 0)
        to[short] ++(0, 4.2)
        node[circ]{} 
    (C2bot)
        node[circ]{}
        to[pC, l=$C_B$] ++(0, 2.1)
        node[circ]{}
        to[pC, l=$C_B$] (C1top)
        node[circ]{}
        ;
        
    \draw [dashed]
    (x.A1) ++(-2.5, 1) coordinate (topleft)
    (x.B1) ++(0.5, 1) coordinate (topright)
    (x.A2) ++(-2.5, -1) coordinate (botleft)
    (x.B2) ++(0.5, -1) coordinate (botright)

    (botleft) -- (topleft) -- (topright) -- (botright) -- (botleft)
    node[midway, anchor=south]{wall wart}
        ;

    \end{circuitikz}
    \bigbreak

\subsection{Peak Input Voltage}
Using a \SI{12}{\volt~RMS} AC input, the peak input voltage will be
%
\begin{displaymath}
  \SI{12}{\volt~RMS} \times \sqrt{2} \approx \SI{17}{\volt}
\end{displaymath}

\subsection{Ripple Voltage}
In a half-wave rectifier, peak-to-peak ripple voltage is given by 
Equation~\eqref{eq:ripple} for a given line frequency $f$, bulk capacitance 
$C_B$, and load current $I$.
\begin{equation}
    \label{eq:ripple}
    V_\mathrm{ripple} = \frac{I}{f \, C_B}
\end{equation}
When unloaded, the output of the rectifier has no ripple. The ripple voltage
increases linearly with load current.
%
The LM317 and LM337 regulators require a minimum load current of
\SI{10}{\milli\ampere}. At this minimum load current, and with a bulk
capacitance of \SI{3300}{\micro\farad}, the ripple voltage will be
%
\begin{displaymath}
  \frac{\SI{0.01}{\ampere}}{(\SI{60}{\hertz})(\SI{3300}{\micro\farad})}
  \approx \SI{51}{\milli\volt}
\end{displaymath}
%
At the supply's maximum load current of \SI{300}{\milli\ampere}, the ripple 
voltage will be
%
\begin{displaymath}
  \frac{\SI{0.3}{\ampere}}{(\SI{60}{\hertz})(\SI{3300}{\micro\farad})}
  \approx \SI{1.5}{\volt}
\end{displaymath}

\subsection{Rectified Output Voltage}
The maximum voltage at the output of the rectification is given by
%
\begin{displaymath}
  V_\mathrm{DC~max} = V_\mathrm{peak} - V_\mathrm{diode~min} = 
  \SI{17}{\volt} - \SI{0.8}{\volt} = \SI{16.2}{\volt}
\end{displaymath}
%
The minimum voltage at the output of the rectification circuit, is given by 
%
\begin{displaymath}
  V_\mathrm{DC~min} = V_\mathrm{peak} - V_\mathrm{diode~max} - 
  V_\mathrm{ripple} = \SI{17}{\volt} - \SI{1.1}{\volt} - \SI{1.5}{\volt} = 
  \SI{14.4}{\volt}
\end{displaymath}
%
The 7812 and 7912 fixed \SI{\pm 12}{\volt} regulators each have a dropout
voltage of \SI{2}{\volt}, and therefore will operate correctly off of a
\SI{14.4}{\volt} supply. The LM317-N and LM337-N regulators have a dropout 
voltage of less than \SI{2}{\volt}, and therefore will also operate
correctly at least up to the supply's maximum rated output of \SI{12}{\volt}.

\section{Power Dissipation}
Maximum power dissipation is with $V_\mathrm{out}$ minimized, i.e.
$V_\mathrm{out} = \SI{1.25}{\volt}$.
%
\begin{displaymath}
  P_\mathrm{max} = I_\mathrm{max} (V_{\mathrm{DC~max}} - V_{o~\mathrm{min}}) = 
  \SI{300}{\milli\ampere} \times (\SI{16.2}{\volt} - \SI{1.25}{\volt}) =
  \SI{4.49}{\watt}
\end{displaymath}
%
The LM317-N and LM337-N thermal parameters (for T0-220 packages) are given in 
Table~\ref{tab:thermal}.
%
\begin{table}
    \centering
    \begin{tabular}{llll}
        \toprule
        & LM317 & LM337 & Unit \\
        \midrule
        $R_{\theta JA}$               & 23.3 & 22.9 & \si{\celsius\per\watt} \\
        $R_{\theta JC\mathrm{(top)}}$ & 16.2 & 15.7 & \si{\celsius\per\watt} \\
        $R_{\theta JB}$               & 4.9  & 4.1  & \si{\celsius\per\watt} \\
        $R_{\theta JC\mathrm{(bot)}}$ & 1.1  & 1.0  & \si{\celsius\per\watt} \\
        \bottomrule
    \end{tabular}
    \caption{Thermal parameters for LM317-N and LM337-N linear regulators in
        the T0-220 package.}
    \label{tab:thermal}
\end{table}
%
Both parts have a maximum operating temperature of \SI{125}{\celsius}. A maximum
design temperature of \SI{100}{\celsius} was selected to give plenty of margin to
the datasheet maximum temperature. With heatsinks installed, the overall thermal
resistance from junction to ambient is the sum of the thermal resistances from
junction to the bottom of the case, the bottom of the case to the heatsink, and
the heatsink to ambient:
%
\begin{displaymath}
    R_{\theta\mathrm{overall}} = 
    R_{\theta JC\mathrm{(bot)}} + R_{\theta C\mathrm{(bot)}S} + R_{\theta SA}
\end{displaymath}
%
The junction temperature is:
%
\begin{displaymath}
    T_J = T_A + R_{\theta\mathrm{overall}} \times P_\mathrm{max}
\end{displaymath}
%
At an ambient temperature $T_A$ of \SI{50}{\celsius}, the maximum
$R_{\theta\mathrm{overall}}$ that gives a junction temperature lower than the
design maximum temperature is \SI{11.15}{\celsius\per\watt}.

Using $R_{\theta C\mathrm{(bot)}S} = \SI{0.5}{\celsius\per\watt}$ (typical for 
a heat conducting pad on a TO-220 package, according to 
\emph{The Art of Electronics}), gives $R_{\theta SA~\mathrm{max}} =
\SI{9.55}{\celsius\per\watt}$ for the LM317 and \SI{9.65}{\celsius\per\watt} for
the LM337. 

\section{Control}
Options
\begin{itemize}
  \item{PID loop in software: encoder sets setpoint, DAC/amp generates adj
          voltage, ADC measures actual output voltage, PID loop corrects}
  \item{hardware feedback loop: DAC/amp generates desired output voltage,
          opamp feedback loop compares output to desired output voltage}
\end{itemize}

\section{Ideas for a future version with more features}
\begin{itemize}
  \item{two mutually isolated primary voltage channels}
  \item{each channel has a variable voltage from \SIrange{0}{15}{\volt}}
  \item{each channel has a variable current limit from \SIrange{0}{1}{\ampere}}
  \item{third output channel for logic? Perhaps with voltage and
      current limits selectable from certain presets, i.e.
      \SIlist{1.8;2.5;3.3;5;12}{\volt}, and
      \SIlist{0.1;0.25;0.5;1.0;2.0}{\ampere}.}
  \item{selectable voltage tracking between two primary channels}
  \item{seven-segment displays for set voltage and current limits for primary
      channels (i.e. four separate displays}
  \item{output enable/disable toggles for each channel}
  \item{output enable LED for each channel}
  \item{current-limit warning LED for each channel}
  \item{separate 10-turn control knobs for each channel's current and voltage
      controls} 
\end{itemize}

\end{document}
